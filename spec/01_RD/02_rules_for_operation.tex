\chapter{運用ルール}
    \section{ロール}

        ユーザーには以下に定義する3つのロールの何れかが割り当てられる。

        \begin{description}
            \item[開発]
                開発側が利用するロール。
                「開発」という名称は便宜上の名称であり、
                実際の意味は「サービスを提供する側」を意味する。
                ロール「開発」はステータス
                    \footnote{ステータスの詳細に関しては\ref{status}ステータスを参照のこと}
                が「起票」、または「回答待ち」のスレッドに対してリプライすることが出来る。
                ロール「開発」が作成したスレッドはステータスが「周知」となる。

            \item[顧客]
                顧客側が利用するロール。
                ロール「開発」とは対になる「サービスを受ける側」を意味する。
                ロール「顧客」はステータスが「回答済み」のスレッドに対してリプライすることが出来る。
                ロール「顧客」が作成したスレッドはステータスが「起票」となる。

            \item[管理]
                開発側のアプリケーション管理者が利用するロール。
                「サービスを提供する側」でかつ、本アプリケーションの管理を行う。
                主にユーザーの作成管理を行うことが出来るが、
                スレッドの作成やリプライをすることはできない。
                閲覧のみは可能である。

        \end{description}

    \section{ステータス\label{status}}

        スレッドのステータスを定義する。

        \subsection{起票}
            顧客ロールによるスレッド作成のフロー。
            顧客ロールに属するユーザーがスレッド作成を行うと、
            スレッドのステータスは「起票」となる。

        \subsection{回答済み}
            開発ロールによるリプライのフロー。
            ステータスが「起票」、もしくは「回答済み」の場合、
            開発ロールに属するユーザーはスレッドに対してリプライが出来る。
            リプライされたスレッドはステータスが「回答済み」となる。

        \subsection{回答待ち}
            顧客ロールによるリプライのフロー。
            ステータスが「回答済み」のスレッドに対して、
            顧客ロールに属するユーザーはリプライできる。
            リプライされたスレッドはステータスが「回答待ち」となる。

        \subsection{完了}
            顧客ロールによる完了リプライのフロー
            顧客ロールによるリプライのフロー。
            ステータスが「回答済み」のスレッドに対して、
            顧客ロールに属するユーザーはリプライが出来る。
            リプライされたスレッドはステータスが「完了」となる。

            ステータスが「完了」となったスレッドは、以後ステータスが変わることは無い。
            つまり、スレッドに対してリプライは出来なくなる。

        \subsection{周知}
            開発ロールによるスレッド作成のフロー。
            開発ロールに属するユーザーがスレッド作成を行うと、
            スレッドのステータスは「周知」となる。

            ステータスが「周知」のスレッドはリプライ不可となる。


