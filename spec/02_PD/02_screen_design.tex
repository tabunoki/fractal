\chapter{画面設計}
    \section{ログイン画面}
        \subsection{概要}
        
ユーザーがログインを行う画面。
ログイン認証をしていないユーザーは、ログイン画面以外の画面にアクセスすると、ログイン画面にリダイレクトされる。
ユーザー名とパスワードの入力を求め、一致するユーザーが存在する場合のみ認証を行う。

        \subsection{ログイン}

ログイン認証を行う。
入力されたユーザー名と、パスワードの組み合わせが登録されているユーザーと一致するか比較する。
一致するユーザーが存在する場合は、一覧画面へと遷移する。
一致するユーザーが存在しない場合は、認証せずに画面上に以下のメッセージを表示する。

\begin{quote}
ユーザー名とパスワードが一致しません。
\end{quote}

    \section{一覧画面}
        \subsection{概要}
作成されたスレッドの一覧を表示する。

            \subsubsection{見出しの表示}
以下の見出しを<h1>タグで表示する。

\begin{quote}
スレッド一覧
\end{quote}

            \subsubsection{一覧の表示項目}


        \subsection{スレッド詳細画面への遷移}
スレッド名に設定されたハイパーリンクをクリックしたとき、
当該のスレッドIDを引数にして、スレッド詳細画面へ遷移する。

        \subsection{ページネーション}
スレッド一覧画面はページネーションを行う。
一画面上に表示するスレッドの件数は20件とする。
ページネーションは先頭ページ、最終ページ、前ページ、次ページ、
現在表示しているページ±5ページに対して、任意にアクセスできる。
ナビゲーションは一覧画面の下部、フッターの上部に表示する。

    \section{作成画面}
        \subsection{概要}
        \subsection{起票}
    \section{詳細画面}
        \subsection{概要}
        \subsection{リプライ}
        \subsection{削除}
    \section{設定画面}
        \subsection{概要}
        \subsection{パスワード変更}
    \section{管理画面}
        \subsection{概要}
        \subsection{ユーザーの作成}
        \subsection{ユーザーの削除}
        \subsection{ユーザーの有効化}
        \subsection{ユーザーの無効化}


